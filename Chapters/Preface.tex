
\chapter*{Abstract} \label{Abstract}
\addcontentsline{toc}{chapter}{Abstract}

Thin targets, in form of foils, stripes or wires, are widely used 
in beam instrumentation to measure various beam parameters, such as intensity, position and size. All these monitors can differ in geometry and material. 
Depending on beam parameters such as intensity, energy, transverse and longitudinal size, the detector can suffer thermomechanical stresses. This can potentially perturb the measurement accuracy and degrade the integrity of the detector. The core of this contribution presents the development of a finite-difference model, developed to simulate the particle-detector interactions and predict the detector material heating and damage. 

This work has been mainly performed in the context of LINAC4, which is the first accelerator at CERN's accelerator chain. Some other facilities (CERN PS Booster, CERN SPS, GSI facility) will also be mentioned in this document. However, the majority of the studies and conclusions will refer to the energy range (45 keV - 160 MeV) and detector types available at LINAC4 (SEM grids and wire scanners). 

To understand the purpose, functionality and limitation of thin target detectors, this document introduces the basic principles of transverse beam dynamics and beam matter interactions, focusing on processes such as: energy deposition, secondary electron emission (SEE), electron backscattering, etc. The thermal model implemented during this work (PyTT) to simulate the thermal evolution of thin target detectors is presented and detailed discussed. To assess the reliability of the simulated results, an experiment performed at LINAC4 is presented. This experiment heavily relied on the theory of thermionic emission to indirectly measure the temperature of the detectors during operation. 

The performance of the simulation tool is discussed, in terms of results sensitivity to parameter uncertainty. Uncertainties in material parameters, such as the emissivity, yield uncertainties in simulation results. To improve our knowledge of the emissivity values of thin tungsten wires, an experimental setup, based on the calorimetric method, was implemented for this work. The experimental setup, emissivity calculation and results are detailed described. 

Some examples of how this work has been useful for CERN operations, and other facilities (like GSI) are presented in these pages. This include, beam power limit calculations for the CERN Linac4 and SPS diagnostics, beam intensity and profile measurements at LINAC4, thin foil detector calibration (H0H- Monitors), SEM grid prototype testing at GSI, etc. 

\chapter*{Acknowledgements} \label{Acknowledgements}
\addcontentsline{toc}{chapter}{Acknowledgements}

If this thesis becomes a reality is because of the kind support and help from many people. I would like to extend my sincere gratitude to all of them. 

There are no proper words to convey my deep gratitude and respect for my thesis
and research advisor, Dr. Federico Roncarolo. I thank him for giving me the opportunity to do this Ph.D., for his continuous support, for his patience, motivation, and his invaluable advice. I would like to express my sincere gratitude to Prof. Francisco Calvino, who since the moment I appeared in his office many years ago, has provided me with guidance, challenging projects and learning opportunities.

Besides my advisors, I would like to thank the rest of the members of my thesis committee: NAME, NAME, NAME, for their encouragement, insightful comments and feedback, which have helped me in completing this project. Special thanks go to, Dr. Ariel Tarifeno. Who has been a huge influence and without whom I would have never been the scientist I am today. 

I am also thankful to all the people that collaborated in the realization of this project. First, I would like to thank Dr. Mariusz Sapinski. This work would not have been possible without his previous studies and knowledge. I thank him for being always willing and enthusiastic to assist in any way he could throughout the research project. Thank you to Miguel Martin, a colleague and good friend, without whom I would (genuinely) not have been able to perform the emittance measurements. His incredible technical skills, his kindness, and his well-timed jokes made the working process so much more entertaining. 

I would like to thank the ABT group, Chiara Bracco and Elisabeth Renner, for their patience and assistance during the long calibration measurements. I would also like to thank the beam instrumentation team at GSI, as well as the members of the PROACTIVE company, for giving me the opportunity to join them in their measurements campaign. 

I would like to thank all the members of the PM section at CERN. I have learned so much from all of you. All those precious moments in corridors, lunchtimes, and various coffee breaks have meant more to me than you know. A special thank you goes to my amazing officemates, Daniele Butti and Luana Parsons, for all those very welcomed distractions. And to my great friends, M. Checcetto, M. Pagin and Jose Carlos, I am not going to say anything, you guys know why your name is here. 

Lastly, I guess I could not finish without a chiche thank you to my mother. M'agradaria donar-te les gràcies per tot el teu suport, el teu carinyo, i l'amor que mas donat sempre. M'agradaria agrair-te tots els teus esforços, tot el que has sacrificat per mi. Sé que no ha sigut fàcil, però jo sempre he sigut molt feliç i si he arribat fins aquí es graciés tu. Intento dir-t'ho tant com puc, però t'ho torno a dir, i ara per escrit i tot. Moltes gràcies mare!

\tableofcontents
\addcontentsline{toc}{chapter}{Contents}

\listoffigures
\addcontentsline{toc}{chapter}{List of Figures}

% ----------- INClude All the chapters ---------- %

\chapter*{Overview} \label{Overview}
\addcontentsline{toc}{chapter}{Overview}

This document has been divided into seven chapters, each one of them trying to cover a distinct topic. This overview is designed to give an insight into the layout of this work and briefly describe the contents of each chapter.

Chapter \ref{ch:Introduction} introduces the context of this work. Given that the majority of the studies presented in this thesis took place at CERN, this first chapter describes with some detail CERN accelerator's chain (Section \ref{sec:CERN_acc_complex}). Paying special attention to LINAC4 (Section \ref{sec:LINAC4}). Some basic concepts about accelerator physics are also introduced (Section \ref{sec:AccPhysPrinc}). Specifically, the concepts of transverse beam physics and transverse beam profiles. The second part of the chapter (Section \ref{sec:BeamDiag}) moves onto beam instrumentation used to measure the previously described concepts. Beam Instrumentation is a very broad topic, so only the detectors relevant to this work are described. Those being, SEM grids (Subsection \ref{sec:SEMgrids}), Wire Scanners (Subsection \ref{sec:WireScan}), Beam Current Transformers (Subsection \ref{sec:BCT}) and Faraday cups (Subsection \ref{sec:FC}).  

Chapter \ref{ch:BeamMatterInter} introduces the basics of beam matter interactions and current generation in thin target detectors. This chapter does not attempt to provide a thorough, in-depth description of this challenging topic. Its goal is to introduce all the physical topics mentioned in this thesis. Numerous references are given along those pages to allow the reader to delve much deeper into the subject if desired. The first part of the chapter focuses on how charged particles interact with matter (Section \ref{sec:BeamMatterInter}). The main concepts introduced are: Energy Loss (Subsection \ref{sec:Bethe}), Multiple scattering and backscattering (Subsection \ref{sus:Scatt}), Path and Range (Subsection \ref{sec:Range}). Afterward, the concept of Secondary Electron Emission (SEE) is introduced (Section \ref{sec:SEY}). 

Chapter \ref{ch:CurrentModeling}) uses all the concepts explained in the previous chapter to present a semi-empirical model used to describe the signal generation in thin target detectors with applications in beam particle accelerators. This model has been extensively used to predict and interpret the signal generated in various detectors, to showcase its applicability an example of signal generation studies at Linac4 is presented (Section \ref{sec:SignalStudiesL4}). The final part of the chapter presents the results obtained in the first profile and current measurements for the LBE run (Section \ref{sec:LBE}). 

The core part of this thesis consisted of the implementation of a program able to simulate the thermal evolution of thin target detectors for applications in particle accelerators, this program has been called PyTT. Chapter \ref{ch:TempModeling} starts with a small explanation about why is it important to implement such a simulation tool (Section \ref{sec:Motivation}). Simulating the thermal evolution implies solving the heat equation (Section \ref{sec:HeatEq}). In our case the terms that have been considered are: Beam Heating (Section \ref{sec:BeamHeating}), Radiative Cooling (Section \ref{sec:RadiativeCooling}), Conduction Cooling (Section \ref{sec:ConductionCooling}), Thermionic Cooling (Section \ref{sec:ThermionicCooling}) and Sublimation Cooling (Section \ref{sec:SublimationCooling}). All these terms are explained in detail in the chapter. The heat equation has to be solved numerically. In this chapter, the numerical tools employed to solve such an equation are also briefly discussed (Section \ref{sec:NumericalMethod}). A graphical user interface was implemented to facilitate the usage of this code. This GUI is briefly discussed and presented in Section \ref{sec:GUI}. The performance of the simulation tool is then discussed, in terms of results sensitivity to parameter uncertainties (Section \ref{sec:ModelUnc})

There are two very distinct sections in Chapter \ref{ch:ThermoMeasur}. Two examples of benchmarking the PyTT program are provided in the first section of the chapter. In Section \ref{sec:ThAtLINAC4} an experiment performed in LINAC4 is described. The goal of the experiment was to compare the simulations of SEM grid detector results with experimental measurements. Here we present the whole experimental process: experimental planning obtained data and simulations-measurement comparison. In Section \ref{sec:AnsysComparison} thermal simulation results obtained with the PyTT code are compared to results obtained with ANSYS, a commercially available software, commonly used and highly trusted for thermo-mechanical studies. 

The second part of Chapter \ref{ch:ThermoMeasur} aims to show how the PyTT code has been used at CERN. For that, two application examples are shown. Those are: Beam power limit calculations, in terms of beam intensity, pulse lenght and beam size, at LINAC4 (Section \ref{sec:BeamPowerL4}).  Beam power limit calculations, in terms of beam size, charges per pulse and wire speed, for fast wire scanners at the SPS (Section \ref{sec:BeamPowerSPS})

When reaching chapter \ref{ch:EmissivityMeas} hopefully it will be clear that uncertainties in material parameters, such as the emissivity of the material, yield big uncertainties in the simulated thermal results. Chapter \ref{ch:EmissivityMeas} describes an experiment, based on the calorimetric method (Section \ref{sec:CalMeth}), to measure the emissivity of thin tungsten wires. The experimental set-up was designed and implemented from scratch explicitellty for this work, and it is described in Section \ref{sec:ExpSetup}. In this chapter, the intermediate steps necessary to obtain the final value of the emissivity are presented. These include: Calibration of the electronics (Section \ref{sec:ElecCal}), Temperature Measurements (Section \ref{sec:IntTemp}), Boundary condition measurements (Section \ref{sec:BoundCond}). The chapter closes with the results of the emissivity measurements for the tungsten wires used at CERN  LINAC4 (Section \ref{sec:EmissRes}).

Chapter \ref{ch:H0Hm} is slightly different from the other chapters of the thesis, and it could almost be taken separately. This chapter describes the calibration of the \hzhm current monitors. These detectors have as an objective to measure the stripping inefficiency of the newly installed CEI system in the PS Booster. Section \ref{sec:CEI} introduces the concept of CEI and focuses on the injection of LINAC4 into the PS Booster. The \hzhm current monitors are then introduced (Section \ref{sec:H0HmDet}).

To fully grasp the complexity of the data analysis performed for the calibration of the detectors, a brief description of the electronics is provided (Section \ref{sec:H0HmElectr}). The various signals to analyze are also described (Section \ref{sec:H0HmSignals}). This is then followed by a detailed description of the process followed for the calibration (Section \ref{sec:H0HmCalproc}), it presents the final results of the calibration of the detectors (Section \ref{sec:H0HmResults}) and some dependencies are discussed (Section \ref{sec:H0Hmdep}). After the calibration these detectors have been extensively used. Section \ref{sec:StrippingIneefMeas} shows an example of measurements taken with these detectors after the calibration. Finally, this chapter finishes presenting some of the current challenges faced by the \hzhm monitors (Section \ref{sec:NoiseProblem}). 

Chapter \ref{ch:GSIMEasurements} presents the results obtained during an experimental campaign at GSI. During these experiments, two SEM grid prototypes, designed and manufactured by PROACTIVE, were tested. In this chapter, a description of the GSI facility is given (Section \ref{sec:FAIR}). This is followed by a summary of the motivations and the experimental conditions in which the Grids were tested  (Section \ref{sec:GSIexpCond}). The results of the experiment are divided in three separate sections. First the dependency of the measured current with the applied voltage is presented (Section \ref{sec:BiasVoltage}). This is followed by a study on the transverse beam profile measurements and its time dependency (Section \ref{sec:ProfileMeas}). In paralel to the measurement thermal simulations were done with the PyTT code, predicting the wire temperature at variuous experimental conditions. The comparison between this predictions and the measured results is presented in Section \ref{sec:GSIThStudies}.

The conclusions succinctly summarize the results obtained in this document and discuss the relevance and impact this study had on the field.